% This file was converted to LaTeX by Writer2LaTeX ver. 1.0.2
% see http://writer2latex.sourceforge.net for more info
\documentclass[12pt]{article}
\usepackage[utf8]{inputenc}
\usepackage[T1]{fontenc}
\usepackage[english]{babel}
\usepackage{amsmath}
\usepackage{amssymb,amsfonts,textcomp}
\usepackage{array}
\usepackage{hhline}
\usepackage{hyperref}
\hypersetup{colorlinks=true, linkcolor=blue, citecolor=blue, filecolor=blue, urlcolor=blue}
\raggedbottom
% Paragraph styles
\renewcommand\familydefault{\rmdefault}
\newenvironment{styleStandard}{\setlength\leftskip{0cm}\setlength\rightskip{0cm}\setlength\parindent{0cm}\setlength\parfillskip{0pt plus 1fil}\setlength\parskip{0cm plus 1pt}\writerlistparindent\writerlistleftskip\leavevmode\normalfont\normalsize\writerlistlabel\ignorespaces}{\unskip\vspace{0cm plus 1pt}\par}
% List styles
\newcommand\writerlistleftskip{}
\newcommand\writerlistparindent{}
\newcommand\writerlistlabel{}
\newcommand\writerlistremovelabel{\aftergroup\let\aftergroup\writerlistparindent\aftergroup\relax\aftergroup\let\aftergroup\writerlistlabel\aftergroup\relax}
\title{}
\author{}
\date{2016-01-28T09:20:57.739684375}
\begin{document}
\begin{styleStandard}
References
\end{styleStandard}


\begin{styleStandard}
Bieder, R. (1995). Native American communities in Wisconsin, 1600-1960: A study of tradition and change. Madison: University of Wisconsin Press. 
\end{styleStandard}


\begin{styleStandard}
Daniels, P. (1990). Fundamentals of grammatology, Journal of the American Oriental Society 119 (4), 727-731. JSTOR 602899
\end{styleStandard}


\begin{styleStandard}
Fletcher, A. (1890a). A phonetic alphabet used by the Winnebago tribe of Indians. Journal of American Folk-Lore, 3(11), 299-301. 
\end{styleStandard}


\begin{styleStandard}
Fletcher, A. (1890b.) The phonetic alphabet of the Winnebago Indians. Proceedings of the American Association for the Advancement of Science, 38th Meeting. Held at Toronto, Ont., August 1889. Salem, Mass: 354-357. 
\end{styleStandard}


\begin{styleStandard}
Helmbrecht, J., and Lehman, C. (Eds.) (2010). Hoc{\textbackslash}k\{a\}k teaching materials (Vol. I). Albany: State University of New York Press. 
\end{styleStandard}


\begin{styleStandard}
Jones, W. (1906). An Algonquian syllabary. In B. Lanfer (Ed.), Boas anniversary volume: anthropological papers written in honor of Franz Boas (pp. 88-93). New York: G. E. Stechert. 
\end{styleStandard}


\begin{styleStandard}
Lipkind, W. (1945). Winnebago grammar. Morningside Heights, NY: Kings?s Crown Press.
\end{styleStandard}


\begin{styleStandard}
Mark, J. (1988). A stranger in her native land: Alice Fletcher and the American Indians. Lincoln: University of Nebraska Press. 
\end{styleStandard}


\begin{styleStandard}
Mesquakie-Sauk (Sac and Fox). (1998-2015). Native languages of the Americas. Retrieved from http://www.native-languages.org/meskwaki-sauk.htm.
\end{styleStandard}


\begin{styleStandard}
Radin, P. (1954). The evolution of an American prose epic: a study in comparative literature (Pt. 1). Basil, Switzerland: Ethnographical Museum. 
\end{styleStandard}


\begin{styleStandard}
Susman, A. L. (1939). The Winnebago syllabary. (Manuscript, Freeman guide no. 3903, Library of the American Philosophical Society, Philadelphia.) 
\end{styleStandard}


\begin{styleStandard}
Susman, A. L. (1943). The accentual system of Winnebago. (Doctoral dissertation, Columbia University, New York.) 
\end{styleStandard}


\begin{styleStandard}
Walker, W. (1981). Native American writing systems. In C. A. Ferguson {\textbackslash}\& S. B. Heath (Eds.), Languages in the USA (pp. 145-174). Cambridge and New York: Cambridge University Press. 
\end{styleStandard}


\begin{styleStandard}
Walker, W. (1996). Native writing systems. In I. Goddard (Vol. Ed.), Handbook of North American Indians (Vol. 17) (pp. 158-184). Washington, D.C.: Smithsonian Institution. 
\end{styleStandard}

\end{document}
